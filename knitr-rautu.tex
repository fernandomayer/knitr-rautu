\documentclass{beamer}\usepackage[]{graphicx}\usepackage[]{color}
%% maxwidth is the original width if it is less than linewidth
%% otherwise use linewidth (to make sure the graphics do not exceed the margin)
\makeatletter
\def\maxwidth{ %
  \ifdim\Gin@nat@width>\linewidth
    \linewidth
  \else
    \Gin@nat@width
  \fi
}
\makeatother

\definecolor{fgcolor}{rgb}{0.345, 0.345, 0.345}
\newcommand{\hlnum}[1]{\textcolor[rgb]{0.686,0.059,0.569}{#1}}%
\newcommand{\hlstr}[1]{\textcolor[rgb]{0.192,0.494,0.8}{#1}}%
\newcommand{\hlcom}[1]{\textcolor[rgb]{0.678,0.584,0.686}{\textit{#1}}}%
\newcommand{\hlopt}[1]{\textcolor[rgb]{0,0,0}{#1}}%
\newcommand{\hlstd}[1]{\textcolor[rgb]{0.345,0.345,0.345}{#1}}%
\newcommand{\hlkwa}[1]{\textcolor[rgb]{0.161,0.373,0.58}{\textbf{#1}}}%
\newcommand{\hlkwb}[1]{\textcolor[rgb]{0.69,0.353,0.396}{#1}}%
\newcommand{\hlkwc}[1]{\textcolor[rgb]{0.333,0.667,0.333}{#1}}%
\newcommand{\hlkwd}[1]{\textcolor[rgb]{0.737,0.353,0.396}{\textbf{#1}}}%

\usepackage{framed}
\makeatletter
\newenvironment{kframe}{%
 \def\at@end@of@kframe{}%
 \ifinner\ifhmode%
  \def\at@end@of@kframe{\end{minipage}}%
  \begin{minipage}{\columnwidth}%
 \fi\fi%
 \def\FrameCommand##1{\hskip\@totalleftmargin \hskip-\fboxsep
 \colorbox{shadecolor}{##1}\hskip-\fboxsep
     % There is no \\@totalrightmargin, so:
     \hskip-\linewidth \hskip-\@totalleftmargin \hskip\columnwidth}%
 \MakeFramed {\advance\hsize-\width
   \@totalleftmargin\z@ \linewidth\hsize
   \@setminipage}}%
 {\par\unskip\endMakeFramed%
 \at@end@of@kframe}
\makeatother

\definecolor{shadecolor}{rgb}{.97, .97, .97}
\definecolor{messagecolor}{rgb}{0, 0, 0}
\definecolor{warningcolor}{rgb}{1, 0, 1}
\definecolor{errorcolor}{rgb}{1, 0, 0}
\newenvironment{knitrout}{}{} % an empty environment to be redefined in TeX

\usepackage{alltt}
\usepackage[utf8]{inputenc}
\usepackage[T1]{fontenc}
\usepackage[brazil]{babel}
\usepackage{graphicx}
\usepackage{verbatim}
\usepackage{alltt}
\usepackage{here}
\usepackage{xcolor}
\usepackage{fancyhdr}
\usepackage{setspace}
\usepackage{indentfirst}
\usepackage{multirow}
\usepackage{makeidx}
\usepackage{wrapfig}
\usepackage[all]{xy}
\usepackage{fancybox}
\usepackage{rotating}
\usepackage{eso-pic}
\usepackage{dcolumn}
\usepackage{color}
\usepackage{lscape}
\usepackage{subfigure}
\usepackage{scalefnt}

\newcommand{\ds}{\displaystyle}

\usetheme{CambridgeUS}

\definecolor{ESALQ}{RGB}{50,93,61}
\definecolor{DarkSeaGreen}{RGB}{143,188,143}
\definecolor{PaleGreen}{RGB}{84,139,84}
\setbeamercolor{frametitle}{fg=black,bg=DarkSeaGreen}
\setbeamercolor{structure}{fg=ESALQ}
\setbeamercolor{palette primary}{fg=ESALQ}
\setbeamercolor{palette secondary}{fg=ESALQ}
\setbeamercolor{palette tertiary}{bg=ESALQ}
\setbeamercolor{title}{fg=black,bg=DarkSeaGreen}
\setbeamercolor{block title}{fg=black,bg=DarkSeaGreen}
\setbeamercolor{block title alerted}{bg=DarkSeaGreen}
\makeindex

\usepackage{xspace}
\providecommand{\eg}{\textit{e.g.}\xspace}
\providecommand{\ie}{\textit{i.e.}\xspace}
\providecommand{\R}{\textbf{R}\xspace}
\providecommand{\emacs}{\textbf{Emacs}\xspace}
\providecommand{\knitr}{\textbf{knitr}\xspace}
\providecommand{\rstudio}{\textbf{RStudio}\xspace}

%% Fontes para codigo
\usepackage[scaled]{beramono} % truetype: Bistream Vera Sans Mono
%\usepackage{inconsolata}

\title[Introdução ao $\LaTeX$]{Introdução ao $\LaTeX$}
\author[]{\small Bruna Gabriela Wendpap \\
  Djair Durand Ramalho Frade \\
  Fernando de Pol Mayer \\
  Luiz Ricardo Nakamura \\
  Maria Cristina Martins \\
  Thiago de Paula Oliveira \\
  Thiago Gentil Ramires \\
  Profa. responsável: Dra. Roseli Aparecida Leandro
}
\institute[ESALQ/USP]{Universidade de São Paulo (USP) \\
  Escola Superior de Agricultura ``Luiz de Queiroz" (ESALQ)}
\date[]{04 de Outubro, 2013}

 \AtBeginSection[]
 {
   \begin{frame}
     \frametitle{Plano de aula}
     \tableofcontents[currentsection]
   \end{frame}
 }

 \AtBeginSubsection[]
 {
   \begin{frame}
     \frametitle{Plano de aula}
     \tableofcontents[currentsection,currentsubsection]
   \end{frame}
 }
\IfFileExists{upquote.sty}{\usepackage{upquote}}{}

\begin{document}
\pgfdeclareimage[height=1.4cm]{logo}{esalq}
\logo{\pgfuseimage{logo}}
\frame{\titlepage}




\begin{frame}{Sumário}
\tableofcontents
\end{frame}

\section{Introdução}

\begin{frame}{Introdução}
Teste
\end{frame}

\section{Configuração}

\begin{frame}[fragile]{Configuração}
  \begin{itemize}
  \item O \knitr pode ser utilizado em qualquer editor de texto, mas
    alguns facilitadores são
    \begin{itemize}
    \item \emacs com ESS
    \item \rstudio
    \end{itemize}
  \item Antes de qualquer outra coisa:
\begin{knitrout}\footnotesize
\definecolor{shadecolor}{rgb}{0.969, 0.969, 0.969}\color{fgcolor}\begin{kframe}
\begin{alltt}
\hlkwd{install.packages}\hlstd{(}\hlstr{"knitr"}\hlstd{,} \hlkwc{dependencies} \hlstd{=} \hlnum{TRUE}\hlstd{)}
\end{alltt}
\end{kframe}
\end{knitrout}

  \end{itemize}
\end{frame}

\begin{frame}[fragile]{Configuração}
A ideia é fazer a seguinte sequência:
\begin{enumerate}
\item Criar um arquivo com a extensão \texttt{.Rnw}
\item Inserir o preâmbulo tradicional do \LaTeX{}, texto e código
\item Compilar o arquivo com a função \texttt{knit()} $\rightarrow$ vai
  gerar um arquivo \texttt{.tex}
\item Compilar o arquivo \texttt{.tex} no \TeX{}Maker (ou outros)
  $\rightarrow$ gera o arquivo \texttt{.pdf}
\end{enumerate}
\end{frame}

\begin{frame}[fragile]{Configuração}
Expressões do \R são inseridas normalmente dentro de um ambiente especial
no arquivo \texttt{.Rnw}:
\begin{verbatim}
<<>>=
...
@
\end{verbatim}
\begin{itemize}
\item Toda expressão do \R que estiver dentro deste \textbf{chunk} será
interpretada quando coompilada pelo \knitr, gerando a saída, gráficos,
etc.
\item Para inserir resultados no meio do texto (\textit{inline}) use
  \verb|\Sexpr{}|
\end{itemize}
\end{frame}

\begin{frame}[fragile]{Configuração}
  \begin{block}{Um exemplo mínimo:}
    \footnotesize
\begin{verbatim}
\documentclass{article}
\usepackage[utf8]{inputenc}
\usepackage[T1]{fontenc}
\usepackage[brazil]{babel}

\begin{document}

Definindo a variável aleatória $X$ com distribuição Normal padrão, ou
seja, $X \sim \textrm{N}(0,1)$
<<>>=
set.seed(1)
(x <- rnorm(10))
@
A média desta variável aleatória é 

{\ttfamily\noindent\bfseries\color{errorcolor}{\\Error in mean(x) : object 'x' not found}}. O primeiro valor é
$X_i = 

{\ttfamily\noindent\bfseries\color{errorcolor}{\\Error in eval(expr, envir, enclos) : object 'x' not found}}$.

\end{document}
\end{verbatim}
\end{block}

\end{frame}

\begin{frame}[fragile]
Teste
\begin{knitrout}\footnotesize
\definecolor{shadecolor}{rgb}{1, 1, 1}\color{fgcolor}\begin{kframe}
\begin{verbatim}
> rnorm(10)
 [1]  0.9887  1.5356  0.3532 -1.6897  1.3897  0.2239  2.7323 -1.0115
 [9] -1.5073  0.4062
> runif(10,
+       1, 2)
 [1] 1.240 1.686 1.663 1.535 1.947 1.686 1.581 1.818 1.761 1.611
\end{verbatim}
\end{kframe}
\end{knitrout}

Mais
\begin{knitrout}\footnotesize
\definecolor{shadecolor}{rgb}{0.969, 0.969, 0.969}\color{fgcolor}\begin{kframe}
\begin{alltt}
\hlnum{1} \hlopt{+} \hlnum{1}
\hlkwd{plot}\hlstd{(}\hlnum{1}\hlstd{)}
\end{alltt}
\end{kframe}
\end{knitrout}


\end{frame}

\end{document}

