\documentclass{article}\usepackage[]{graphicx}\usepackage[]{color}
%% maxwidth is the original width if it is less than linewidth
%% otherwise use linewidth (to make sure the graphics do not exceed the margin)
\makeatletter
\def\maxwidth{ %
  \ifdim\Gin@nat@width>\linewidth
    \linewidth
  \else
    \Gin@nat@width
  \fi
}
\makeatother

\definecolor{fgcolor}{rgb}{0.345, 0.345, 0.345}
\newcommand{\hlnum}[1]{\textcolor[rgb]{0.686,0.059,0.569}{#1}}%
\newcommand{\hlstr}[1]{\textcolor[rgb]{0.192,0.494,0.8}{#1}}%
\newcommand{\hlcom}[1]{\textcolor[rgb]{0.678,0.584,0.686}{\textit{#1}}}%
\newcommand{\hlopt}[1]{\textcolor[rgb]{0,0,0}{#1}}%
\newcommand{\hlstd}[1]{\textcolor[rgb]{0.345,0.345,0.345}{#1}}%
\newcommand{\hlkwa}[1]{\textcolor[rgb]{0.161,0.373,0.58}{\textbf{#1}}}%
\newcommand{\hlkwb}[1]{\textcolor[rgb]{0.69,0.353,0.396}{#1}}%
\newcommand{\hlkwc}[1]{\textcolor[rgb]{0.333,0.667,0.333}{#1}}%
\newcommand{\hlkwd}[1]{\textcolor[rgb]{0.737,0.353,0.396}{\textbf{#1}}}%

\usepackage{framed}
\makeatletter
\newenvironment{kframe}{%
 \def\at@end@of@kframe{}%
 \ifinner\ifhmode%
  \def\at@end@of@kframe{\end{minipage}}%
  \begin{minipage}{\columnwidth}%
 \fi\fi%
 \def\FrameCommand##1{\hskip\@totalleftmargin \hskip-\fboxsep
 \colorbox{shadecolor}{##1}\hskip-\fboxsep
     % There is no \\@totalrightmargin, so:
     \hskip-\linewidth \hskip-\@totalleftmargin \hskip\columnwidth}%
 \MakeFramed {\advance\hsize-\width
   \@totalleftmargin\z@ \linewidth\hsize
   \@setminipage}}%
 {\par\unskip\endMakeFramed%
 \at@end@of@kframe}
\makeatother

\definecolor{shadecolor}{rgb}{.97, .97, .97}
\definecolor{messagecolor}{rgb}{0, 0, 0}
\definecolor{warningcolor}{rgb}{1, 0, 1}
\definecolor{errorcolor}{rgb}{1, 0, 0}
\newenvironment{knitrout}{}{} % an empty environment to be redefined in TeX

\usepackage{alltt}
\usepackage[utf8]{inputenc}
\usepackage[T1]{fontenc}
\usepackage[brazil]{babel}
\usepackage[margin=2.5cm]{geometry}
%% Uma fonte alternativa para o documento geral
% \usepackage{pslatex}
%% Fonte alternativa para códigos
% \usepackage{inconsolata}

\title{Introdução ao uso do knitr}
\author{Muitos Autores}
\date{04 de Setembro, 2013}
\IfFileExists{upquote.sty}{\usepackage{upquote}}{}

\begin{document}
\maketitle





Definindo a variável aleatória $X$ com distribuição Normal padrão, ou
seja, $X \sim \textrm{N}(0,1)$
\begin{knitrout}
\definecolor{shadecolor}{rgb}{0.969, 0.969, 0.969}\color{fgcolor}\begin{kframe}
\begin{alltt}
\hlkwd{set.seed}\hlstd{(}\hlnum{1}\hlstd{)}
\hlstd{(x} \hlkwb{<-} \hlkwd{rnorm}\hlstd{(}\hlnum{10}\hlstd{))}
\end{alltt}
\begin{verbatim}
##  [1] -0.6265  0.1836 -0.8356  1.5953  0.3295 -0.8205  0.4874  0.7383
##  [9]  0.5758 -0.3054
\end{verbatim}
\end{kframe}
\end{knitrout}

A média desta variável aleatória é 0.1322. O primeiro valor é
$X_1 = -0.6265$.

\begin{knitrout}
\definecolor{shadecolor}{rgb}{0.969, 0.969, 0.969}\color{fgcolor}\begin{kframe}
\begin{alltt}
\hlkwd{rnorm}\hlstd{(}\hlnum{10}\hlstd{)}
\end{alltt}
\end{kframe}
\end{knitrout}


\begin{knitrout}
\definecolor{shadecolor}{rgb}{0.969, 0.969, 0.969}\color{fgcolor}\begin{kframe}
\begin{alltt}
\hlkwd{set.seed}\hlstd{(}\hlnum{1}\hlstd{)}
\hlkwd{rbeta}\hlstd{(}\hlnum{10}\hlstd{,} \hlnum{2}\hlstd{,} \hlnum{5}\hlstd{)}
\end{alltt}
\begin{verbatim}
##  [1] 0.1755 0.3243 0.1456 0.3570 0.1477 0.3944 0.4582 0.2280 0.6757 0.3710
##  [1] -0.62124 -2.21470  1.12493 -0.04493 -0.01619  0.94384  0.82122
##  [8]  0.59390  0.91898  0.78214
\end{verbatim}
\begin{alltt}
\hlkwd{rgamma}\hlstd{(}\hlnum{10}\hlstd{,} \hlnum{2}\hlstd{,} \hlnum{5}\hlstd{)}
\end{alltt}
\begin{verbatim}
##  [1] 0.31854 0.77153 0.62490 0.15462 0.19431 0.24504 0.27535 0.16200
##  [9] 0.05751 0.59225
\end{verbatim}
\end{kframe}
\end{knitrout}





\newpage

Testando mais opções de chunks.

\begin{knitrout}
\definecolor{shadecolor}{rgb}{0.969, 0.969, 0.969}\color{fgcolor}\begin{kframe}
\begin{alltt}
\hlstd{> }\hlkwd{rnorm}\hlstd{(}\hlnum{10}\hlstd{)}
\end{alltt}
\begin{verbatim}
 [1]  1.15654  0.83205 -0.22733  0.26614 -0.37670  2.44136 -0.79534
 [8] -0.05488  0.25014  0.61824
\end{verbatim}
\end{kframe}
\end{knitrout}


\begin{knitrout}\large
\definecolor{shadecolor}{rgb}{0.969, 0.969, 0.969}\color{fgcolor}\begin{kframe}
\begin{verbatim}
rnorm(10)
##  [1] -0.17262 -2.22390 -1.26361  0.35873 -0.01105 -0.94065 -0.11583
##  [8] -0.81497  0.24226 -1.42510
\end{verbatim}
\end{kframe}
\end{knitrout}


\begin{knitrout}
\definecolor{shadecolor}{rgb}{0.969, 0.969, 0.969}\color{fgcolor}\begin{kframe}
\begin{alltt}
\hlkwd{rnorm}\hlstd{(}\hlnum{10}\hlstd{,} \hlnum{10}\hlstd{,} \hlnum{5}\hlstd{)}
\end{alltt}
\begin{verbatim}
##  [1] 11.830 11.242 10.326 10.096 11.287  6.755  9.404 13.321 15.505 10.719
\end{verbatim}
\end{kframe}
\end{knitrout}


\begin{knitrout}
\definecolor{shadecolor}{rgb}{0.969, 0.969, 0.969}\color{fgcolor}\begin{kframe}
\begin{alltt}
\hlkwd{rnorm}\hlstd{(}\hlnum{10}\hlstd{,}
      \hlnum{10}\hlstd{,} \hlnum{5}\hlstd{)}
\end{alltt}
\begin{verbatim}
##  [1]  9.411  5.440  2.812  6.015 16.270 13.861  8.902  7.876  7.905 14.985
\end{verbatim}
\end{kframe}
\end{knitrout}


\newpage

Trabalhando com tabelas.

\begin{kframe}
\begin{alltt}
\hlcom{## Carrega o pacote}
\hlkwd{require}\hlstd{(xtable,} \hlkwc{quietly} \hlstd{=} \hlnum{TRUE}\hlstd{)}
\hlcom{## Tira uma amostra de 10 linhas da base de dados Iris}
\hlstd{am} \hlkwb{<-} \hlkwd{sample}\hlstd{(}\hlnum{1}\hlopt{:}\hlkwd{nrow}\hlstd{(iris),} \hlkwc{size} \hlstd{=} \hlnum{10}\hlstd{)}
\hlstd{iris.am} \hlkwb{<-} \hlstd{iris[am, ]}
\end{alltt}
\end{kframe}% latex table generated in R 3.0.2 by xtable 1.7-1 package
% Fri Oct  4 17:49:52 2013
\begin{table}[ht]
\centering
\begin{tabular}{rrrrrl}
  \hline
 & Sepal.Length & Sepal.Width & Petal.Length & Petal.Width & Species \\ 
  \hline
59 & 6.60 & 2.90 & 4.60 & 1.30 & versicolor \\ 
  57 & 6.30 & 3.30 & 4.70 & 1.60 & versicolor \\ 
  133 & 6.40 & 2.80 & 5.60 & 2.20 & virginica \\ 
  95 & 5.60 & 2.70 & 4.20 & 1.30 & versicolor \\ 
  109 & 6.70 & 2.50 & 5.80 & 1.80 & virginica \\ 
  88 & 6.30 & 2.30 & 4.40 & 1.30 & versicolor \\ 
  131 & 7.40 & 2.80 & 6.10 & 1.90 & virginica \\ 
  43 & 4.40 & 3.20 & 1.30 & 0.20 & setosa \\ 
  28 & 5.20 & 3.50 & 1.50 & 0.20 & setosa \\ 
  125 & 6.70 & 3.30 & 5.70 & 2.10 & virginica \\ 
   \hline
\end{tabular}
\end{table}



\newpage

A tabela abaixo é a tabela de número \ref{tab:iris}.

% latex table generated in R 3.0.2 by xtable 1.7-1 package
% Fri Oct  4 17:49:52 2013
\begin{table}[ht]
\centering
\begin{tabular}{rrrrrl}
  \hline
 & Sepal.Length & Sepal.Width & Petal.Length & Petal.Width & Species \\ 
  \hline
76 & 6.60 & 3.00 & 4.40 & 1.40 & versicolor \\ 
  131 & 7.40 & 2.80 & 6.10 & 1.90 & virginica \\ 
  29 & 5.20 & 3.40 & 1.40 & 0.20 & setosa \\ 
  112 & 6.40 & 2.70 & 5.30 & 1.90 & virginica \\ 
  106 & 7.60 & 3.00 & 6.60 & 2.10 & virginica \\ 
  137 & 6.30 & 3.40 & 5.60 & 2.40 & virginica \\ 
  79 & 6.00 & 2.90 & 4.50 & 1.50 & versicolor \\ 
  102 & 5.80 & 2.70 & 5.10 & 1.90 & virginica \\ 
  56 & 5.70 & 2.80 & 4.50 & 1.30 & versicolor \\ 
  15 & 5.80 & 4.00 & 1.20 & 0.20 & setosa \\ 
   \hline
\end{tabular}
\caption{Uma legenda para a tabela.} 
\label{tab:iris}
\end{table}



\newpage

Essa é a tabela \ref{tab:iris2} com legenda em cima.

% latex table generated in R 3.0.2 by xtable 1.7-1 package
% Fri Oct  4 17:49:52 2013
\begin{table}[ht]
\centering
\caption{Uma legenda para a tabela.} 
\label{tab:iris2}
\begin{tabular}{rrrrrl}
  \hline
 & Sepal.Length & Sepal.Width & Petal.Length & Petal.Width & Species \\ 
  \hline
140 & 6.90 & 3.10 & 5.40 & 2.10 & virginica \\ 
  43 & 4.40 & 3.20 & 1.30 & 0.20 & setosa \\ 
  88 & 6.30 & 2.30 & 4.40 & 1.30 & versicolor \\ 
  17 & 5.40 & 3.90 & 1.30 & 0.40 & setosa \\ 
  123 & 7.70 & 2.80 & 6.70 & 2.00 & virginica \\ 
  47 & 5.10 & 3.80 & 1.60 & 0.20 & setosa \\ 
  113 & 6.80 & 3.00 & 5.50 & 2.10 & virginica \\ 
  39 & 4.40 & 3.00 & 1.30 & 0.20 & setosa \\ 
  32 & 5.40 & 3.40 & 1.50 & 0.40 & setosa \\ 
  73 & 6.30 & 2.50 & 4.90 & 1.50 & versicolor \\ 
   \hline
\end{tabular}
\end{table}



\newpage

Sem nomes de linhas.

% latex table generated in R 3.0.2 by xtable 1.7-1 package
% Fri Oct  4 17:49:52 2013
\begin{table}[ht]
\centering
\caption{Uma legenda para a tabela.} 
\label{tab:iris3}
\begin{tabular}{rrrrl}
  \hline
Sepal.Length & Sepal.Width & Petal.Length & Petal.Width & Species \\ 
  \hline
5.00 & 3.50 & 1.30 & 0.30 & setosa \\ 
  5.00 & 3.40 & 1.60 & 0.40 & setosa \\ 
  6.80 & 2.80 & 4.80 & 1.40 & versicolor \\ 
  5.80 & 2.70 & 3.90 & 1.20 & versicolor \\ 
  5.70 & 3.80 & 1.70 & 0.30 & setosa \\ 
  4.90 & 3.60 & 1.40 & 0.10 & setosa \\ 
  6.30 & 2.90 & 5.60 & 1.80 & virginica \\ 
  6.40 & 3.10 & 5.50 & 1.80 & virginica \\ 
  5.80 & 4.00 & 1.20 & 0.20 & setosa \\ 
  7.30 & 2.90 & 6.30 & 1.80 & virginica \\ 
   \hline
\end{tabular}
\end{table}



\newpage

Com a saída de um modelo linear.

\begin{kframe}
\begin{alltt}
\hlstd{mod} \hlkwb{<-} \hlkwd{lm}\hlstd{(Petal.Length} \hlopt{~} \hlstd{Petal.Width, iris)}
\end{alltt}
\end{kframe}% latex table generated in R 3.0.2 by xtable 1.7-1 package
% Fri Oct  4 17:49:52 2013
\begin{table}[ht]
\centering
\begin{tabular}{rrrrr}
  \hline
 & Estimate & Std. Error & t value & Pr($>$$|$t$|$) \\ 
  \hline
(Intercept) & 1.0836 & 0.0730 & 14.85 & 0.0000 \\ 
  Petal.Width & 2.2299 & 0.0514 & 43.39 & 0.0000 \\ 
   \hline
\end{tabular}
\end{table}



\newpage

Figuras.

\begin{knitrout}
\definecolor{shadecolor}{rgb}{0.969, 0.969, 0.969}\color{fgcolor}\begin{kframe}
\begin{alltt}
\hlkwd{plot}\hlstd{(iris)}
\end{alltt}
\end{kframe}
\includegraphics[width=\maxwidth]{figure/fig1} 

\end{knitrout}


\newpage

\begin{knitrout}
\definecolor{shadecolor}{rgb}{0.969, 0.969, 0.969}\color{fgcolor}\begin{kframe}
\begin{alltt}
\hlkwd{plot}\hlstd{(iris)}
\end{alltt}
\end{kframe}
\includegraphics[width=.5\linewidth]{figure/fig2} 

\end{knitrout}


\newpage

\begin{knitrout}
\definecolor{shadecolor}{rgb}{0.969, 0.969, 0.969}\color{fgcolor}\begin{kframe}
\begin{alltt}
\hlkwd{plot}\hlstd{(iris[,} \hlopt{-}\hlnum{5}\hlstd{])}
\end{alltt}
\end{kframe}

{\centering \includegraphics[width=.5\linewidth]{figure/fig3} 

}



\end{knitrout}


\newpage

Algum texto sobre a figura \ref{fig:fig4}.

\begin{knitrout}
\definecolor{shadecolor}{rgb}{0.969, 0.969, 0.969}\color{fgcolor}\begin{kframe}
\begin{alltt}
\hlkwd{plot}\hlstd{(iris)}
\end{alltt}
\end{kframe}\begin{figure}[!htb]


{\centering \includegraphics[width=.5\linewidth]{figure/fig4} 

}

\caption[Legenda da figura]{Legenda da figura.\label{fig:fig4}}
\end{figure}


\end{knitrout}


\newpage

\begin{knitrout}
\definecolor{shadecolor}{rgb}{0.969, 0.969, 0.969}\color{fgcolor}\begin{kframe}
\begin{alltt}
\hlkwd{plot}\hlstd{(Petal.Length} \hlopt{~} \hlstd{Petal.Width, iris)}
\end{alltt}
\end{kframe}

{\centering \includegraphics[width=.5\linewidth]{figure/fig51} 

}


\begin{kframe}\begin{alltt}
\hlkwd{plot}\hlstd{(Sepal.Length} \hlopt{~} \hlstd{Petal.Length, iris)}
\end{alltt}
\end{kframe}

{\centering \includegraphics[width=.5\linewidth]{figure/fig52} 

}



\end{knitrout}


\newpage

\begin{knitrout}
\definecolor{shadecolor}{rgb}{0.969, 0.969, 0.969}\color{fgcolor}\begin{kframe}
\begin{alltt}
\hlkwd{plot}\hlstd{(Petal.Length} \hlopt{~} \hlstd{Petal.Width, iris)}
\hlkwd{plot}\hlstd{(Sepal.Length} \hlopt{~} \hlstd{Petal.Length, iris)}
\end{alltt}
\end{kframe}

{\centering \includegraphics[width=.45\linewidth]{figure/fig61} 
\includegraphics[width=.45\linewidth]{figure/fig62} 

}



\end{knitrout}









\end{document}
